\documentclass{article}
\usepackage[utf8]{inputenc}
\usepackage[T1]{fontenc}
\usepackage{mathtools}
\newcommand\Myperm[2][n]{\prescript{#1\mkern-2.5mu}{}P_{#2}}
\newcommand\Mycomb[2][n]{\prescript{#1\mkern-0.5mu}{}C_{#2}}
\begin{document}
~\\
CS 334 Database System ~\\
ps01 ~\\\\
1. A primary key is a key in a relation that is unique for each tuple. If a student has more than one advisor, then there will be more than one tuples in the relation with the same "s\_id" which violates the definition of "primary key". Instead of using "s\_id" as the primary key, using a pair of attributes consisting of \{s\_id, i\_id \} will be a better option as it is unique for each tuple.
~\\\\
2. Assuming every attribute and tuple is unique. It's basically multiplying permutation of the attributes and permutation of the tuples to calculate the ways. \par
a) Multiplying the permutation of 3 and 3:\par
Ways = 3! $\cdot$ 3! = 6 $\cdot$ 6 = 36 \par
b) Multiplying the permutation of 4 and 5:\par
Ways = 4! $\cdot$ 5! = 24 $\cdot$ 120 = 2880 \par
c) Multiplying the permutation of n and m:\par
Ways = n! $\cdot$ m!
~\\\\
3.\par
(a) The attribute "producerCertNum" in relation "Movie" should be a foreign key, referencing attribute "certNum" in relation "MovieExec". \par
(b) The attribute \{ movieTitle, movieYear \} in relation "StarsIn" should be a foreign key, referencing attribute \{ title, year \} in relation "Movie". \par
(c) The attribute "starName" in relation "StarsIn" should be a foreign key, referencing attribute "name" in relation "MovieStar". \par
(d) It's not possible to be done as a foreign key constraint. The foreign key in "Movie" must refer to the primary key in "StarsIn". Even if \{ title, year \} can refer to \{ movieTitle, movieYear\}. However, \{ movieTitle, movieYear\} does not form a primary key in relation "StarsIn" for the fact there could be more than one star in a movie. And in fact, the primary key of "StarsIn" contains three attributes \{ movieTitle, movieYear, starName\}. Therefore, it's not possible.

\pagebreak
4. \par
\begin{table}[h]
\centering
\caption{artist\_credit relation}
\label{my-label}
\begin{tabular}{|c|c|c|}
\hline
\multicolumn{3}{|c|}{artist\_credit}                                             \\ \hline
id   & name                                                      & artist\_count \\ \hline
1001 & Queen \& David Bowie                                      & 2             \\ \hline
1002 & Jean-Michel Jarre                                         & 1             \\ \hline
1003 & Tracy W. Bush, Derek Duke, Jason Hayes and Glenn Stafford & 4             \\ \hline
\end{tabular}
\end{table}

\par

\begin{table}[h]
\centering
\caption{artist\_credit\_name relation}
\label{my-label}
\begin{tabular}{|c|c|c|c|c|}
\hline
\multicolumn{5}{|c|}{artist\_credit\_name}                            \\ \hline
artist\_credit & position & artist & name              & join\_phrase \\ \hline
1001           & 1        & 101    & Queen             & \&           \\ \hline
1001           & 2        & 102    & David Bowie       & Null         \\ \hline
1002           & 1        & 103    & Jean-Michel Jarre & Null         \\ \hline
1003           & 1        & 104    & Tracy W. Bush     & ,            \\ \hline
1003           & 2        & 105    & Derek Duke        & ,            \\ \hline
1003           & 3        & 106    & Jason Hayes       & and          \\ \hline
1003           & 4        & 107    & Glenn Stafford    & Null         \\ \hline
\end{tabular}
\end{table}

\par
\begin{table}[h]
\centering
\caption{artist relation}
\label{my-label}
\begin{tabular}{|c|c|}
\hline
\multicolumn{2}{|c|}{artist} \\ \hline
id     & name                \\ \hline
101    & Queen               \\ \hline
102    & David Bowie         \\ \hline
103    & Jean Michel Jarre   \\ \hline
104    & Tracy W. Bush       \\ \hline
105    & Derek Duke          \\ \hline
106    & Jason Hayes         \\ \hline
107    & Glenn Stafford      \\ \hline
\end{tabular}
\end{table}
\end{document}

